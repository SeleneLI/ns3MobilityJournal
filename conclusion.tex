% -< SECTION
% >--------------------------------------------------------------------
\section{Conclusion and future work}\label{sec:conclusion}
As a promising technology for the future Internet architecture, LISP attracts
more and more attention \ed{BD: ref?}. There exist some LISP implementations,
but they do not support LISP-MN or they are proprietary. Further, although measurements on
LISP-testbeds can provide real time performance, due to the complicated
topological structure, it is somewhat like a black box test which hinders us to
find the exact explanation for some results. This highlights the importance to
have an open source simulator for LISP in particular to support LISP-MN
functionality. In this paper, we present our implementation for LISP/LISP-MN
within ns-3, since the latter is a largely accepted simulator in networking
research. The simulation results show that our implementation works well, and
reveal the current LISP-MN proposal with a double encapsulation that has an high
level delay during handover procedure. Our simulator can be a perfect choice to
test the improvements of LISP-MN. 

There are two possible directions to support IP mobility in LISP: host-based
(i.e. LISP-MN) and network-based (i.e., xTR) mobility. We can compare the
performance between LISP double encapsulation described in this paper with only
host supporting LISP and only router supporting LISP leveraging our proposed
simulator. As Map-Versioning~\cite{rfc6834} is another Mapping Cache update
mechanism, we can also compare the performance between it and SMR that we
present in this paper by our simulator.

\ed{BD: this is a (very) short paper. I would suggest to shorten the conclusion}

