%-< SECTION >--------------------------------------------------------------------
\section{Introduction}\label{sec:intro}

The \dfn{Locator/Identifier Separation Protocol} (LISP)~\cite{rfc6830} is
initially proposed to solve the scalability and flexibility issues of current
Internet architecture, since it decouples the \dfn{Routing Locators} (RLOCs,
i.e., an attachment point in the Internet topology) and \dfn{Endpoint
Identifiers} (EIDs, i.e., a communicating end-point). This allows the BGP
routing tables in the Internet core to only announce the globally routable RLOCs
whereas EIDs are only locally used within the LISP-sites. LISP is under
standardization at the IETF for about ten years and, with time, more and more
advantages are found such as: seamless mobility at terminal or in the Data
Center~\cite{saucez2016locator}, IPv6 transition and traffic
engineering~\cite{saucez2012designing}. In particular, the continuous
communication without interruption during the handover of terminals becomes a
hot topic recently~\cite{phoomikiattisak2016control}.

To test the various features of LISP, two LISP testbeds are built: LISP Beta
Network~\cite{lispbeta} and LISP-Lab platform~\cite{lisplab}. Unfortunately, both of
them only implement the basic LISP architecture and do not support mobility
functionality. Some platform-specific LISP implementations, such as
OpenLISP~\cite{phung2014openlisp}, Open Overlay Router (OOR)~\cite{LISPmob}, and
Cisco's implementation~\cite{CiscoIOS}, can provide realistic LISP evaluations
but lack the required flexibility for testing new and advanced LISP features.
% Further, a simulating environment for LISP would be valuable for researches as it would allow them to test, in a controlled and reproducible environment, the tomorrow LISP.
Thus, a simulating environment for LISP would be necessary and valuable for researches to improve LISP.
\ed{BD: better but should probably be rewritten a little bit.}\yue{Modified, better?} On
one hand, LISP has already been implemented in
ns-3~\cite{lezama2009implementation}, a widely used open-source simulator by
academic researchers and educational.  However, this implementation does not
support mobility.  On the other hand, LISP and mobility features are already
implemented in OMNET++~\cite{klein2012integration}. However, the source code is
not available online. To promote the development of LISP and as ns-3 gradually
shows the potential to take the momentum in network simulator
domain~\cite{rana2017implementation}, it highlights the importance to implement
LISP and its mobility extensions under ns-3. % and motivates our work.  
\ed{BD: last sentence unclear}\yue{Simplified, better?}

In this paper, we introduce the implementation of basic LISP functions and
mobility features on ns-3 by both modifying the existent ns-3 modules and
integrating new ones. Our implementation is validated through a mobility
scenario in which a terminal changes its attachment points while keeping the
communication up with the remote node. All the handover procedures are
transparent to the terminal.  However, the delay to receive the packets is high.

\ed{BD: you mention that the Omnet implementation is not open-source.  What
about yours?  If it is open-source,please provide a link in the paper to the
repo.}

The rest of the paper is organized as follows: Sec.~\ref{sec:lisp} reviews LISP
architecture, highlighting the LISP mobility mechanism; Sec.~\ref{sec:sim_model}
analyzes the design and implementation of our prototype, and afterwards,
Sec.~\ref{sec:evaluation} presents preliminary evaluation results of our
implementations. Sec.~\ref{sec:conclusion} concludes this paper by summarizing
its main achievements and discussing potential future works.
