%-< SECTION >--------------------------------------------------------------------
\section{LISP Overview}\label{sec:lisp}


%-< SUB SECTION >--------------------------------------------------------------------
\subsection{Basic Architecture}\label{sec:archi}

LISP splits the current IP addressing space into two sub-spaces. One called EID
indicating the host identifier is locally used within a LISP-site. The other
one, called RLOC, represents the attachment point in the topology, i.e., the
border router, and is globally routable between LISP-sites on the Internet core.
The binding between EID and RLOC is called \dfn{mapping information} and is
stored in the \dfn{Mapping Distribution System} (MDS) consisting of \dfn{Map
Resolver} (MR) and \dfn{Map Server} (MS)~\cite{rfc6833}. The border router is in
charge of searching for the mapping information either in its own Cache or in
the MDS.  It then encapsulates the conventional IP packets into LISP packets and
forwards them out of the LISP-site on the Internet core.  This router is called
\dfn{Ingress Tunnel Router} (ITR).  On the other hand, routers receiving the
LISP packets and decapsulating them into traditional packets for the LISP-site
behind them are named \dfn{Egress Tunnel Router} (ETR). One usually refers to
both of them as \dfn{xTR}.


%-< SUB SECTION >--------------------------------------------------------------------
\subsection{LISP Mobile Node}\label{sec:lispMN}

The LISP Mobile Node (LISP-MN) implements a lightweight version of the standard xTR
functionality~\cite{mn14} supporting a node itself fast and seamless roaming% and being discovered in an efficient and scalable manner 
\ed{BD: unclear}\yue{Better?}. It can directly interact
with the MR to get the mapping information associated to the remote host, to which it communicates \ed{BD:
which remote host?}\yue{Is it clear now?}.
When a LISP-MN resides in a LISP-site, it is assigned an EID taken from the
site's EID-prefix as its RLOC, called \emph{Local RLOC} (\emph{LRLOC}) \ed{BD: I
think I understand the sentence but this is unclear}.
Thus, LISP-MN stores not only its permanent unique EID but also the LRLOC
and registers this mapping information to MS. The conventional IP packets
produced by the LISP-MN are encapsulated on itself using the permanent EID as
the inner source address and Local RLOC as the outer source address 
The encapsulated LISP packets are forwarded to xTR and encapsulated again by the
method of basic encapsulation of xTR. The procedure of packets being
encapsulated two times are called \emph{double encapsulation}.  \ed{BD:
this 1st encapsulation is unclear to me.  The EDI and LRLOC are used to
forward the packet to the xTR.  Then, the xTR encapsulates towards the distant
sites.  How this xTR knows the mapping of the distant site?  It is unclear
from this paragraph}.

%-< SUB SECTION >--------------------------------------------------------------------
\subsection{Mapping Cache Update Mechanisms}\label{sec:updateMechanisms}

When the ETRs change their LISP databases \ed{BD: never heard about the
databases before}, the only way that remote ITR can get the updated mapping
information is to re-request the mapping. Thus, Solicit-Map-Request (SMR) is
proposed to get the latest mapping.  \ed{BD: the broken English here makes the
sentence unclear.  SMR cannot be ``proposed''.}

Soliciting a Map-Request is used by ETRs to tell remote ITRs to update the
mappings they have cached in the Control Plane~\cite{rfc6830}. When the mappings
in the Database of ETR change, the ETR sends the Map-Requests with the SMR bit
set (called SMR message) for each RLOC in its Cache. A remote ITR that receives
the SMR message will send a Map-Request to the MR. During the sending SMR
message and receiving the new Map-Reply, the ITR continues to use the mapping
information previously cached, and that may cause the packet loss.

\ed{Bd: globally, this last subsection is unclear.}
