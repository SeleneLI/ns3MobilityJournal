%-----------------------------------------------------------------
\begin{abstract}
The \emph{Locator/Identifier Separation Protocol} (LISP) reconstructs the
current IP addressing space to improve the scalability and no interrupt mobility
issues. LISP Mobile Node (LISP-MN) is based on the basic LISP functionality to
provide seamless mobility across networks. The basic LISP architecture is
deployed on LISP Beta Network and LISP-Lab platform to offer the researchers a
realistic experimental environment, but both do not support LISP-MN. Some
simulation models with LISP extensions are implemented on various simulators,
but are unfortunately not open source. Providing a free and flexible simulation
model with the basic LISP architecture as well as the extensions so to help
researchers quickly test new LISP behaviors motivates our work. This paper
introduces the implementation of the basic LISP architecture model and LISP-MN
in ns-3. It also provides the evaluation results in mobility
scenario to validate the model and shows that, when the current proposal of
LISP-MN is behind a LISP-site, it has a very high delay during the handover
procedure.
\end{abstract}
%-----------------------------------------------------------------
